%%%%%%%%%%%%%%%%%%%%%%%%%%%%%%%%%%%%%%%%%
% Friggeri Resume/CV
% XeLaTeX Template
% Version 1.0 (5/5/13)
%
% This template has been downloaded from:
% http://www.LaTeXTemplates.com
%
% Original author:
% Adrien Friggeri (adrien@friggeri.net)
% https://github.com/afriggeri/CV
%
% License:
% CC BY-NC-SA 3.0 (http://creativecommons.org/licenses/by-nc-sa/3.0/)
%
% Important notes:
% This template needs to be compiled with XeLaTeX and the bibliography, if used,
% needs to be compiled with biber rather than bibtex.
%
%%%%%%%%%%%%%%%%%%%%%%%%%%%%%%%%%%%%%%%%%

\documentclass[]{friggeri-cv} % Add 'print' as an option into the square bracket to remove colors from this template for printing

\usepackage[most]{tcolorbox}
\usepackage {multicol}
\usepackage{pgffor}

\newtcbox{\badge}[1][blue]{
  on line, 
  arc=2pt,
  colback=#1!50!black,
  colframe=#1!50!black,
  fontupper=\color{white},%,
  boxrule=1pt, 
  boxsep=0pt,
  left=6pt,
  right=6pt,
  top=1pt,
  bottom=2pt
}

\def\containerdata{
{Containers},
{Docker},
{Kubernetes},
{OpenShift},
{Operators},
{Go},
{Cloud Native},
{Helm},
{GitOps},
}

\def\cddata{
{ArgoCD},
{Terraform},
{Infrastructure as Code},
{CI/CD},
{Vault},
{Travis CI},
{Ansible},
{Jenkins},
{Git},
{Github},
{IBM Cloud},
{Artifactory},
{Python},
{SQL},
{NoSQL},
{PostgreSQL},
{CouchDB},
{Redis}
}

\def\extradata{
{Node JS},
{React JS},
{React Native},
{Redux},
{Jest},
{Mocha},
{Enzyme},
{JavaScript},
{HTML/CSS}
}

\begin{document}

\header
{Jacob}{Wernette}
{Site Reliability Engineer}
{werne2j@gmail.com\space\space\space\space\space\space github.com/werne2j \space\space\space\space\space werne2j.medium.com}

%----------------------------------------------------------------------------------------
%	SIDEBAR SECTION
%----------------------------------------------------------------------------------------


%----------------------------------------------------------------------------------------
%	EDUCATION SECTION
%----------------------------------------------------------------------------------------

\section{Education}

\begin{entrylist}
%------------------------------------------------
\entry
{December 2014}
{B.S in Computer Science{\normalfont}}
{Central Michigan University}

%------------------------------------------------
\end{entrylist}

%----------------------------------------------------------------------------------------
%	WORK EXPERIENCE SECTION
%----------------------------------------------------------------------------------------

\section{Professional Experience}

\begin{entrylist}
%------------------------------------------------
\entry
{2020--Present}
{IBM}
{Raleigh, North Carolina}
{\emph{Site Reliability Engineer}
    \begin{itemize}
        \item Initiated transformation of traditional Operations team into a Site Reliability Team
        \item SME for CI/CD, Containerization, Cloud Native and SRE principals across organization
        \item Architected and implemented new OpenShift cluster layout while increasing resiliency
        \item Transitioned team from manual infrastructure to IaC with Terraform and ArgoCD
        \item Increased observability of clusters, improved alerting and monitoring, reduced toil for engineers
    \end{itemize}}\\

\entry
{2018--2020}
{Red Hat}
{Raleigh, North Carolina}
{\emph{Software Engineer - CI/CD}
 \begin{itemize}
        \item Automated manual build, test and release processes of productization of Open Source tools
        \item Built test suite for Jenkins Shared Library and became a SME on the subject, giving presentations and writing four articles
        \item Planned and Built a Command Line tool using Go to replace Jenkins Shared Library internals
    \end{itemize}}\\

\entry
{2015--2018}
{IBM}
{Raleigh, North Carolina}
{\emph{Software Engineer II (Sept 2017 - Aug 2018)} \\
\emph{Software Engineer I (Jan 2015 - Sept 2017)}
\begin{itemize}
        \item Designed, developed, tested, and supported tools used in IBM's internal business from proof of concept through production
        \item Built Node JS REST APIs, web applications, React JS component libraries, and NPM packages
    \end{itemize}}\\

%------------------------------------------------
\end{entrylist}

%----------------------------------------------------------------------------------------
%	PROJECTS SECTION
%----------------------------------------------------------------------------------------

\section{Open Source Projects}

\begin{entrylist}
%------------------------------------------------
\entry
{2020--Present}
{argocd-vault-plugin}
{https://github.com/IBM/argocd-vault-plugin}
{An ArgoCD plugin to retrieve secrets from Hashicorp Vault and inject them into Kubernetes secrets}\\

%------------------------------------------------
\end{entrylist}

%----------------------------------------------------------------------------------------
%	TECHNOLOGY SECTION
%----------------------------------------------------------------------------------------
\section{Technologies}
\foreach \x in \containerdata
{
   \foreach \y in \x
   {
     \badge[lightgreen]{\y}
   }  
}
\foreach \x in \cddata
{
   \foreach \y in \x
   {
     \badge[green]{\y}
   }  
}
\foreach \x in \extradata
{
   \foreach \y in \x
   {
     \badge[forestgreen]{\y}
   }  
}



%{\badge{Containers}, Docker, Kubernetes, OpenShift, Operators, Go, Cloud Native, Helm} \\
%{ArgoCD, Terraform, Infrastructure as Code, CI/CD, Vault, Travis CI, Ansible, Jenkins} \\
%{Git, Github, IBM Cloud, Artifactory, Jira, Python, SQL, NoSQL, PostgreSQL, CouchDB, Redis} \\
%{Node JS, React JS, React Native, Redux, Jest, Mocha, Enzyme, JS/HTML/CSS, iOS Development (Objective-C/Swift)} \\

\header
{Jacob}{Wernette}
{Site Reliability Engineer}
{werne2j@gmail.com\space\space\space\space\space\space github.com/werne2j \space\space\space\space\space werne2j.medium.com}


%----------------------------------------------------------------------------------------
%	ARTICLES SECTION
%----------------------------------------------------------------------------------------

\section{Articles}

\begin{entrylist}
%------------------------------------------------
\entry
{02/2021}
{Solving ArgoCD Secret Management with the argocd-vault-plugin}
{}
{https://itnext.io/argocd-secret-management-with-argocd-vault-plugin-539f104aff05}\\

\entry
{08/2019}
{What Are Jenkins Shared Libraries And Why You Should Use Them}
{}
{https://medium.com/@werne2j/jenkins-shared-libraries-part-1-5ba3d072536a}\\

\entry
{09/2019}
{How To Build Your Own Jenkins Shared Library}
{}
{https://medium.com/@werne2j/how-to-build-your-own-jenkins-shared-library-9dc129db260c}\\

\entry
{10/2019}
{Unit Testing a Jenkins Shared Library}
{}
{https://medium.com/@werne2j/unit-testing-a-jenkins-shared-library-9bfb6b599748}\\

\entry
{11/2019}
{Collecting Code Coverage for a Jenkins Shared Library}
{}
{https://medium.com/@werne2j/collecting-code-coverage-for-a-jenkins-shared-library-c2d8f502732e}\\

%------------------------------------------------
\end{entrylist}

\end{document}
